\chapter{FlowVR}
\label{FlowVR}

This version of \parflow{} can be used as a FlowVR module in FlowVR workflows
\cite{dreherflexible2014}
permitting
In Situ and In Transit File I/O, analysis, monitoring and steering of running simulations.
See \cite{thesisFriedemann2018} for further background information.
The motivation for In Situ and In Transit in high performance computation is mainly to avoid
blocking file I/O by either completely avoiding file I/O (In Situ/ In Transit analysis,
monitoring and steering) or performing it in parallel to the numerical calculations.
As seen in \cite{thesisFriedemann2018} already In Situ file output for \parflow{} gives performance
boost especially
for dump intensive problems on multiple nodes of a high performance computing platform.
In Situ and In Transit analysis, monitoring and steering allow completely new scientific
workflows already useful on small domains but becoming necessary on the continental problem
scale.

In the following we show how to make use of \parflow{} in a FlowVR workflow with the tools
provided in this \parflow{} distribution. These tools build up a framework denoted as
parFlowVR. This allows to run \parflow{} and other tools of the parFlowVR ecosystem
as a distributed parallel application.

\section{Installation}
\subsection{Dependencies}
When installing the parFlowVR framework you will need to install
FlowVR first.
It can be retrieved from \\\code{https://gitlab.inria.fr/flowvr/flowvr-ex}%
\footnote{The public access will be opened soon. We recommend using the version
with the git hash \code{628fd3b7348c3fb4e282360f90da1e2636b9e42a} or a newer one.}.
We highly recommend to have Swig (\code{http://www.swig.org}),
VisIt (\code{https://wci.llnl.gov/simulation/computer-codes/visit/}) and
Python with the NumPy and netCDF packages installed too. To learn how to get all those
dependencies have a look in the \code{installer.sh} script of your \parflow{} distribution.

\subsection{Compiling}
The parFlowVR framework is only available in the CMake build system. We recommend to set the following
CMake switches:
\begin{display}\begin{verbatim}
-DPARFLOW_ENABLE_FLOWVR=ON

# Needed for the Python analyzer API
-DBUILD_PYTHON_ANALYZER_API=ON
-DNUMPY_I_PATH=<Directory where to find numpy.i>

# Needed for the visit connector
-DVISIT_LIBRARY_LIBSIM=<Path to libsim (V2)>
-DVISIT_INCLUDE_DIR=<Path to visit include directory>
\end{verbatim}\end{display}

\subsection{The Runtime Environment}
Especially when working on high performance platforms, the parFlowVR framework's runtime
environment must be carefully set up.
In the following we give an example for setting environment variables impacting the
parFlowVR behavior.
\begin{display}\begin{verbatim}
# set this to easily import the Python analyzer API. The python version (here 2.7)
# should match your system's python version.
export PYTHONPATH=$PYTHONPATH:$PARFLOW_DIR/lib/python2.7

# The following line should be in your bashrc file as otherwise FlowVR modules cannot be
# started remotely
export PATH=$PATH:<path to flowvr binaries>

# Load the FlowVR environment:
source flowvr-suite-config.sh

# optional:

# If your high performance platform is using an alternative to ssh for remote shells
# specify it like this:
export FLOWVR_SSH="oarsh"

# In some cases you want to use a different version of parFlowVR modules in your
# distributed applications. Especially on some high performance platforms it might not be
# possible to directly start e.g. ParFlow or the netcdf writer module as for instance
# some modules must be loaded before startup.
# If the following variables are set to point to an executable it is executed instead.
export PARFLOW_EXE=<path to the parflow starter script>
export NETCDFWRITER_EXE<path to the netcdf writer starter script>
\end{verbatim}\end{display}

\section{Setting up a FlowVR workflow}
Normally 3 files, all placed in the same folder, are used to describe a workflow using the parFlowVR framework:
\begin{itemize}
\item a \code{.py} file describing the parFlowVR workflow. Typically it is named
\code{parflowvr.py}.
\item a \code{.tcl} script defining the \parflow{} problem as described in
Section~\ref{Main Input File (.tcl)}
\item a \code{do.sh} file. If executed it runs the problem as a distributed application.
\end{itemize}
\subsection{The do.sh file}
To run a distributed application using parFlowVR, 4 steps are needed:
\begin{enumerate}
\item create the \parflow{} input database and prepare the input files (\code{tclsh <problemname>.tcl})
\item run the FlowVR Daemon on all participating nodes (\code{flowvrd})
\item create the xml files defining the application (typically a call to \code{python parflowvr.py})
\item run the distributed application (\code{flowvr parflowvr})
\end{enumerate}
All this is abstracted by the \code{do.sh} scripts.
Thus to start a well configured parFlowVR application it is enough to run this script.
Typically the process topology and the \parflow{} problem name are defined in the \code{do.sh} script.

There are prepared scripts for multi core (\code{\$PARFLOW\_DIR/bin/parflowvr/doMPI.sh})
and single core \\(\code{\$PARFLOW\_DIR/bin/parflowvr/do.sh}) parFlowVR applications.
To use them the following constraints need to be fulfilled:
\begin{itemize}
\item the \code{<problemname>.tcl} file produces an input database named
  \code{<problemname>.pfidb} with the \code{pfwritedb} command (This is a classical
    \parflow{} \code{.tcl} script thus it defines problems according to
    Section~\ref{Main Input File (.tcl)}.
\item the \code{.py} script is named \code{parflowvr.py}
\item \code{parflowvr.py} has to
  accept the following parameters: \code{<problemname> <P> <Q> <R>}
\item in case of multi core the \code{.tcl} file needs to accept the following parameters:
  \code{<P> <Q> <R>}
\end{itemize}
  \code{<P> <Q> <R>} denote the process topology as in Section~\ref{Computing Topology}.
\subsection{The .py File}
To write these files ground up, a knowledge of the FlowVR middleware and the modules backing
parFlowVR is needed. This can be obtained from the FlowVR documentation, especially from
the chapters on the FlowVR-Appy (see \code{http://flowvr.sourceforge.net/FlowVRDoc.html})
and \cite{thesisFriedemann2018}. The FlowVR-Appy is the framework to define distributed
FlowVR applications data flows in Python

In contrast we propose to adapt the \code{parflowvr.py} files from the examples given in the following
and from the test cases given in \code{flowvr/testcases}.
\section{In Situ File Write}
To benefit from In Situ file write, a \parflow{} module's out port needs to be connected to
a netcdf writer module's in port. This is done by the following \code{parflowvr.py} script:

\begin{display}\begin{verbatim}
import sys, os

from filters import *

parflow_dir = os.getenv('PARFLOW_DIR')
sys.path.append(parflow_dir + '/bin/parflowvr')
from parFlowVR_modules import *

# We are working with a hostlist to support multi node too.
problemName, P, Q, R, hostlist = sys.argv[1:6]
hosts = hostlist.split(',')


# Add all your out ports as in FlowVR.Outports.Names here that one wants to dump
# Remember to not add two outports dumping the same variable as this leads to undefined
# behavior.
outports = ["pressure", "saturation"]

# here: one writer over all...1
parflowmpi = ParflowMPI(','.join(hosts[1:]),
        problemName=problemName,
        outports=outports) # ports as specified in tcl file

netcdfwriter = NetCDFWriter("netcdfwriter", host=hosts[0])

all_ports=[]
for portname in outports:
  all_ports += parflowmpi.getPort(portname)

treePressure = generateNto1(prefix="comNto1PressureMerge", in_ports=all_ports
  , arity = 2)
treePressure.link(netcdfwriter.getPort("in"))

app.generate_xml("parflowvr")
\end{verbatim}\end{display}

To our problem's \code{.tcl} file (named \code{flowvr.tcl}) we add:
\begin{display}\begin{verbatim}
[...]
pfset Process.Topology.P        [lindex $argv 0]
pfset Process.Topology.Q        [lindex $argv 1]
pfset Process.Topology.R        [lindex $argv 2]

pfset FlowVR True
pfset FlowVR.OnEnd         SendEmpty
pfset FlowVR.Outports.Names "pressure saturation"
pfset FlowVR.Outports.pressure.Periodicity 1
pfset FlowVR.Outports.pressure.Variable "pressure"
pfset FlowVR.Outports.pressure.Offset 0
pfset FlowVR.Outports.saturation.Periodicity 1
pfset FlowVR.Outports.saturation.Variable "saturation"
pfset FlowVR.Outports.saturation.Offset 0

# instead of pfrun:
pfwritedb flowvr
\end{verbatim}\end{display}
You can add even more out ports to the \code{.tcl} file dumping other variables. Remember
to connect them all to the same netCDF writer module to write all the variables in the same netCDF file.
This is done by adding them to the \code{outports} list in \code{parflowvr.py}. Furthermore it is important to
give them all the same periodicity and offset as the merge tree we are using awaits messages
on all input ports at the same time to begin its work.
\footnote{Other workflows e.g. dumping different variables with different periodicities
are possible but would need big changes in the \code{parflowvr.py} script.}

And the \code{do.sh} contains:
\begin{display}\begin{verbatim}
P=`echo $(cat $MACHINEFILE | wc -l) -1 | bc`
Q=1
R=1
PROBLEMNAME=flowvr

# run it:
$HOME/bin/froggy_doMPI.sh $PROBLEMNAME $P $Q $R
\end{verbatim}\end{display}
Where the \code{MACHINEFILE} environment variable is the filepath of a file containing
all participating nodes formatted like this (here: 3 cores per node. Such a file is
normally retrieved from the job scheduler on high performance platforms):
\begin{display}\begin{verbatim}
host1
host1
host1
host2
host2
host2
.
.
.
\end{verbatim}\end{display}

\section{In Situ Analysis and Steering}
There is an API to write analysis in C or in Python. The C API gives more performance but you will
need to understand the FlowVR C-API (fca) too and writing analyzers in C is very error
prone. As shown in
\cite{thesisFriedemann2018} even the Python analyzer API
(provided by the \code{pypfAnalyzer}-Python module) runs performant and thus is recommended to
use for creating analyzers. To use the Python analyzer API \parflow{} must have been compiled with
the \code{-DBUILD\_PYTHON\_ANALYZER\_API:BOOL=ON} switch.
How to write own analyzers can be easily understood regarding the following example. We construct a
\code{analyzer.py} script that is in the same folder as the other project files with the
following content:
\begin{display}\begin{verbatim}
#!/usr/bin/python2 -u
import pypfAnalyzer as pfa
import numpy as np

# If receiving a merged grid message (this typically happens when using more than one
# ParFlow instances dumping data to a single analyzer module through merge filters)
# onGridMessage is called for every part! Thus ix, iy and iz are given in the
# GridMessageMetadata object m.
# They point to the starting position of the grid message's grid in the domain.
# If you need to react to the whole grid at once you can either use a
# MergeGridMessages filter or cache the array in python.
def onGridMessage(arr, m):
    operand = np.ones(shape = arr.shape) * 42.
    for k in range(arr.shape[0]):
        for j in range(arr.shape[1]):
            for i in range(arr.shape[2]):
                # The coordinates x, y and z are grid coordinates in the problem domain
                x = i + m.ix;
                y = j + m.iy;
                z = k + m.iz;

                # Doing sample analysis and calculate the operand
                # Attention: the coordinates are not in the obvious order!
                operand[k,j,i] += x + y + z
    # Steer the simulation with the operand.
    # Depending on the FlowVR graph it is sometimes not guaranteed that the Steer
    # is performed for the next Simulation step already.
    pfa.SendSteer(pfa.ACTION_SET, pfa.VARIABLE_PRESSURE, m.ix, m.iy, m.iz, operand)

# Set our onGridMessage function as listener to react to incoming grid messages containing
# a variable dump from the parflow simulation
pfa.SetGridMessageParser(onGridMessage)

# Run as a FlowVR module
pfa.run([])
\end{verbatim}\end{display}

The \code{SendSteer} command allows to induce a set
(\code{ACTION\_SET}), add (\code{ACTION\_ADD}) or multiplication
(\code{ACTION\_MULTIPLY}) on the simulations data grid of one of the variables
\begin{display}\begin{verbatim}
  VARIABLE_PRESSURE
  VARIABLE_SATURATION
  VARIABLE_POROSITY
  VARIABLE_MANNING
  VARIABLE_PERMEABILITY_X
  VARIABLE_PERMEABILITY_Y
  VARIABLE_PERMEABILITY_Z
\end{verbatim}\end{display}
The steer action can also be performed on only a part of the variable grid held by \parflow{}.
Thus the parameters \code{ix}, \code{iy} and \code{iz} are used to give the starting point.
More information on this topic can be found in the documentation comments especially in
\code{flowvr/pfanalyzer/pypfAnalyzer.h}.

The \code{parflowvr.py} in our example contains:
\begin{display}\begin{verbatim}
from filters import *
import os, sys

parflow_dir = os.getenv('PARFLOW_DIR')
sys.path.append(parflow_dir + '/bin/parflowvr')
from parFlowVR_modules import *

problemName, P, Q, R = sys.argv[1:5]

# Components:
pres = FilterPreSignal("PreSignal", nb=1)

parflowmpi = ParflowMPI(("localhost,"*int(P)*int(Q)*int(R))[:-1],  # cut last ,
        problemName,
        ["out0"])  # ports as specified in tcl file

analyzer = Analyzer("Analyzer", "python -u analyzer.py")

# Connections:
controller = FilterMergeItExt("Controller")
analyzer.getPort("out").link(controller.newInputPort())

# This works as all the parflow instances are synchron
parflowmpi.getPort("endIt")[0].link(pres.getPort("in"))
pres.getPort("out").link(controller.getPort("order"))
controller.getPort("out").link(parflowmpi.getPort("in"))

treePressure = generateNto1(prefix="comNto1PressureMerge",
  in_ports = parflowmpi.getPort("out0"), arity = 2)
treePressure.link(analyzer.getPort("in"))


app.generate_xml("parflowvr")
\end{verbatim}\end{display}

The problems \code{.tcl} file (\code{flowvr.tcl}) contains:
\begin{display}\begin{verbatim}
[...]
pfset Process.Topology.P        [lindex $argv 0]
pfset Process.Topology.Q        [lindex $argv 1]
pfset Process.Topology.R        [lindex $argv 2]

pfset FlowVR True
pfset FlowVR.SteerLogMode "VerySimple"
pfset FlowVR.Outports.Names "out0"

pfset FlowVR.Outports.out0.Periodicity 1
pfset FlowVR.Outports.out0.Variable  "pressure"
pfset FlowVR.Outports.out0.Offset 0

# instead of pfrun:
pfwritedb flowvr
\end{verbatim}\end{display}

And we launch it with the \code{do.sh}:
\begin{display}\begin{verbatim}
P=2
Q=1
R=1
PROBLEMNAME=flowvr
$PARFLOW_DIR/bin/parflowvr/doMPI.sh $PROBLEMNAME $P $Q $R
\end{verbatim}\end{display}

This example runs on one node only. But you can change the \code{<P> <Q> <R>} values in the \code{do.sh}
script to profit from multiple cores.

\section{In Situ Visualization}
For this \parflow{} must be compiled with the \code{-DPARFLOW\_ENABLE\_VISIT\_CONNECTOR:BOOL=ON}
switch and libsim and the libsim include files must be set, for example like this:
\begin{display}\begin{verbatim}
-DVISIT_INCLUDE_DIR:PATH=<path to VisIt>/linux-x86_64/libsim/V2/include
-DVISIT_LIBRARY_LIBSIM:FILEPATH=<path to VisIt>/linux-x86_64/libsim/V2/lib/libsimV2.a
\end{verbatim}\end{display}

If you want to connect to a running simulation with VisIt\footnote{\code{https://wci.llnl.gov/simulation/computer-codes/visit/}}
without intermediate file write
we propose to add a visit connector module to your \code{parflowvr.py} file.
This is done by the following lines (to be inserted before the
\code{app.generate\_xml} statement. Works only with workflows using the \code{FilterMergeItExt}).
\begin{display}\begin{verbatim}
[...]
visit = VisIt("visit-connector")
treeSnap.link(visit.getPort("in"))

# connect to the FilterMergeItExt object to make it assynchronous (do not
# wait for a snapshot request every iteration)
visit.getPort("triggerSnap").link(controller.newInputPort())

app.generate_xml("parflowvr")
\end{verbatim}\end{display}
After starting the parFlowVR application you can open the simulation in VisIt:\\
\code{File > Open File > ~/.visit/simulations/<simulation file>.sim2}. Now add the variable's
chart you want to see and click \code{Draw}. In the current implementation it takes up to 3 time steps of
simulation until you retrieve the data and VisIt begins
to render it. You can request new snapshots of the simulation's state by clicking
\code{trigger snap} under \code{File > Simulations} in VisIt.

See \code{flowvr/testcases/visit} for further reference.

\section{Properly Ending parFlowVR Applications}
\label{section-Properly-ending-parFlowVR-applications}
When setting \code{FlowVR.OnEnd SendEmpty} in the \code{.tcl} file, ParFlow will send an
empty message when all times have been simulated. This way other modules can react to this
properly. At the moment Python Analyzer API modules as well as netcdf writer modules
shut down the whole workflow as soon as they try to process such an empty grid message.

\section{Further Information}
There are many things not yet explained in this version of the documentation.
% for example analyzer and steering with multiple nodes...
Some of them
will be added in later versions but it is still highly recommended, to regard the source codes in e.g.
the \code{flowvr/scripts} \code{flowvr/examples}, \code{flowvr/testcases}
and \code{flowvr/filters} folders as they are well documented and show what is possible with In Situ and In Transit.

Also the developer tools from the FlowVR package for example \code{flowvr-glgraph} to
visualize FlowVR- and thus also parFlowVR workflows are very useful.
