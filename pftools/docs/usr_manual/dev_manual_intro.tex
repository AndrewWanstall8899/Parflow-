%=============================================================================
% Chapter: Introduction
%=============================================================================

\chapter{Introduction}
\label{Introduction}

\parflow{} is a contaminant transport and groundwater flow model
designed to run efficiently in a multi-processor computing environment
on a variety of commercially-available machines.
This has been designed by scientists in the Center for Computational
Sciences and Engineering {\em CCSE} at Lawrence Livermore National
Laboratory, with support from the Earth Science and Environmental
Protection Departments.
Because \parflow{} is able to utilize the computing power that is
available on today's supercomputers, it is able to more accurately model
the effects of subsurface heterogeneity on fluid transport processes.
Heterogeneity is present on many scales in the subsurface, ranging
from the microscopic capillary processes to macroscopic variations in
structure, such as fractures and faults.
\parflow{} attempts to account for heterogeneity and model its
effects on contaminant flow.

\parflow{} is intended to be used as a tool for modeling
environmental transport processes.
A number of aspects of this problem are active areas of research:
numerical algorithms in a parallel environment, geostatistical
methods for characterizing porous media, and the fundamental
physics of multiphase transport in heterogeneous porous media are
some of the major research topics that will be addressed in
constructing \parflow{}.
In order to allow developers to easily reconfigure the \parflow{}
simulator to test new methods for approaching this complex problem,
the software system has been designed in a modular fashion that allows
alternative methods for accomplishing various tasks to be inserted
into the system.
The user has the freedom to choose from among the various options to
configure the simulator and problem setup at run-time in order to
easily compare alternatives.
As additional capability is developed for the \parflow{} simulator,
it must conform to the design standards that have been chosen to allow
seamless integration into the existing system.
This manual is intended to describe the \parflow{} software in sufficient
detail to allow members of the \parflow{} development team to
gain a comprehensive understanding of the \parflow{} system
and some of the issues that have led to the design decisions that
have been made.


