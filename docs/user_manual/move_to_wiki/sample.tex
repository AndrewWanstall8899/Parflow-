
%=============================================================================
%=============================================================================

\section{Running the Sample Problem}
\label{Running the Sample Problem}

Here, we assume that \parflow{} is already built.  The following steps
 will allow you to run a simple test problem supplied with the
 distribution.
\begin{enumerate}

\item
We first create a directory in which to run the problem,
then copy into it some supplied default input files.
So, do the following anywhere in your \file{\$HOME} directory:
\begin{display}\begin{verbatim}
mkdir foo
cd foo
cp $PARFLOW_DIR/examples/default_single.tcl .
chmod 640 *
\end{verbatim}\end{display}
We used the directory name \file{foo} above;
you may use any name you wish\footnote{We use \emph{foo} and \emph{bar} 
just as placeholders for whatever directory you wish you use, 
see also http://en.wikipedia.org/wiki/Foobar}.
The last line changes the permissions of the files so that
you may write to them.

\item
Run \parflow{} using the pftcl file as a TCL script
\begin{display}\begin{verbatim}
tclsh default_single.tcl
\end{verbatim}\end{display}

\end{enumerate}
You have now successfully run a simple \parflow{} problem.
For more information on running \parflow{},
see \S~\ref{Running ParFlow}.

