%=============================================================================
%=============================================================================

\chapter{Getting Started}
\label{Getting Started}

This chapter is an introduction to setting up and running \parflow{}.
In \S~\ref{ParFlow Solvers} we describe the solver options available 
for use with \parflow{} applications.

%=============================================================================
%=============================================================================
\section{ParFlow Solvers}
\label{ParFlow Solvers}

\parflow{} can operate using a number of different solvers.  Two of these solvers, 
IMPES (running in single-phase, fully-saturated mode, not multiphase) and RICHARDS 
(running in variably-saturated mode, not multiphase, with the options of land surface
processes and coupled overland flow) are detailed below. 
This is a brief summary of solver settings used to simulate under three sets of conditions, fully-saturated,
variably-saturated and variably-saturated with overland flow.  A complete, detailed explanation of the 
solver parameters for ParFlow may be found later in this manual.
To simulate fully saturated, steady-state conditions set the solver to IMPES, an example is given below. 
This is also the default solver in ParFlow, so if no solver is specified the code solves using IMPES.

\begin{verbatim}
pfset Solver               Impes
\end{verbatim}

To simulate variably-saturated, transient conditions, using Richards' equation, 
variably/fully saturated, transient with compressible storage set the solver to RICHARDS.  
An example is below.  This is also the solver used to simulate surface flow or coupled 
surface-subsurface flow.

\begin{verbatim}
pfset Solver             Richards
\end{verbatim}

To simulate overland flow, using the kinematic wave approximation to the shallow-wave 
equations, set the solver to RICHARDS and set the upper patch boundary condition for the 
domain geometry to OverlandFlow, an example is below.  This simulates overland flow, independently 
or coupled to Richards' Equation as detailed in \cite{KM06}.  The overland flow boundary 
condition can simulate both uniform and spatially-distributed sources, reading a distribution of
fluxes from a binary file in the latter case. 
\begin{verbatim}
pfset Patch.z-upper.BCPressure.Type	OverlandFlow
\end{verbatim}

For this case, the solver needs to be set to RICHARDS: 
\begin{verbatim}
pfset Solver		Richards
\end{verbatim}

\parflow{} may also be coupled with the land surface model \code{CLM} \cite{Dai03}.  
This version of \code{CLM} has been extensively modified to be called from within \parflow{} 
as a subroutine, to support parallel infrastructure including I/O and most importantly with modified 
physics to support coupled operation to best utilize the integrated hydrology
 in \parflow{} \cite{MM05, KM08a}.  To couple \code{CLM} into \parflow{} first the
 \code{--with-clm} option is needed in the \code{./configure} command as indicated in \S~\ref{Installing ParFlow}. 
 Second, the \code{CLM} module needs to be called from within \parflow{}, this is done using the following 
 solver key:
\begin{verbatim}
pfset Solver.LSM CLM
\end{verbatim}
Note that this key is used to call \code{CLM} from within the nonlinear solver time loop 
and requires that the solver bet set to RICHARDS to work.  Note also that this key defaults to 
\emph{not} call \code{CLM} so if this line is omitted \code{CLM} will not be called from within 
\parflow{} even if compiled and linked.  Currently, \code{CLM} gets some of it's information from \parflow{} 
such as grid, topology and discretization, but also has some of it's own input files for land cover, land cover
types and atmospheric forcing.
