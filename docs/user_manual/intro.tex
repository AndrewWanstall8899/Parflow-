%=============================================================================
% Chapter: Introduction
%=============================================================================

\chapter{Introduction}
\label{Introduction}

\parflow{} (\emph{PARallel FLOW}) is an integrated hydrology 
model that simulates surface and subsurface flow.
\parflow{} \cite{Ashby-Falgout90, Jones-Woodward01, KM06, M13} 
is a parallel simulation platform that operates in three modes:\begin{enumerate}
\item
steady-state saturated; 
\item
variably saturated; 
\item
and integrated-watershed flow.
\end{enumerate}
\parflow{} is especially suitable for large scale problems on a range
of single and multi-processor computing platforms. \parflow{}
simulates saturated and variably saturated
subsurface flow in heterogeneous porous media in three spatial
dimensions using a mulitgrid-preconditioned conjugate gradient solver
\cite{Ashby-Falgout90} and a Newton-Krylov nonlinear solver
\cite{Jones-Woodward01}. \parflow{} has recently been extended to
coupled surface-subsurface flow to enable the simulation of hillslope
runoff and channel routing in a truly integrated fashion
\cite{KM06}. \parflow{} is also fully-coupled with the land surface
model \code{CLM} \cite{Dai03} as described in \cite{MM05,KM08a}.  The
development and application of \parflow{} has been on-going for more
than 20 years \cite{Meyerhoff14a, Meyerhoff14b, Meyerhoff11, Mikkelson13,
RMC10, Shrestha14, SNSMM10, Siirila12a,
Siirila12b, SMPMPK10, Williams11, Williams13, FM10, Keyes13, 
KRM10, Condon13a, Condon13b, M13, KRM10, KRM10, SNSMM10, DMC10, AM10,
MLMSWT10, M10, FM10, KMWSVVS10, SMPMPK10, FFKM09, KCSMMB09, MTK09, dBRM08, 
MK08b, KM08b, KM08a, MK08a, MCT08,MCK07,MWH07,
  KM06, MM05, TMCZPS05, MWT03, Teal02, WGM02, Jones-Woodward01, MCT00,
  TCRM99, TBP99, TFSBA98, Ashby-Falgout90} and resulted in some of the
most advanced numerical solvers and multigrid preconditioners for
massively parallel computer environments that are available
today. Many of the numerical tools developed within the \parflow{}
platform have been turned into or are from libraries that are now
distributed and maintained at LLNL ({\em Hypre} and {\em SUNDIALS},
for example).  An additional advantage of \parflow{} is the use of a
sophisticated octree-space partitioning algorithm to depict complex
structures in three-space, such as topography, different hydrologic
facies, and watershed boundaries. All these components implemented
into \parflow{} enable large scale, high resolution watershed
simulations. 

\parflow{} is primarily written in \emph{C}, uses a modular
architecture and contains a flexible communications layer to
encapsulate parallel process interaction on a range of platforms.
\code{CLM} is fully-integrated into \parflow{} as a module and has
been parallelized (including I/O) and is written in \emph{FORTRAN
  90/95}.  \parflow{} is organized into a main executable
\file{\emph{pfdir}/pfsimulator/parflow_exe} and a library
\file{\emph{pfdir}/pfsimulator/parflow\_lib} (where \file{\emph{pfdir}} is
the main directory location) and is comprised of more than 190
separate source files.  \parflow{} is structured to allow it to be
called from within another application (\emph{e.g.} WRF, the Weather Research 
and Forecasting atmospheric model) or as a
stand-alone application.  There is also a directory structure for the
message-passing layer \file{\emph{pfdir}/pfsimulator/amps} for the
associated tools \file{\emph{pfdir}/pftools} for \code{CLM}
\file{\emph{pfdir}/pfsimulator/clm} and a directory of test cases
\file{\emph{pfdir}/test}.

\section{How to use this manual}
\label{how to}
This manual describes how to use \parflow{}, and is intended for
hydrologists, geoscientists, environmental scientists and engineers. 
This manual is written assuming the reader has a basic understanding
of Linux / UNIX environments, how to compose and execute scripts in various 
programming languages (e.g. TCL), and is familiar with groundwater and 
surface water hydrology, parallel computing, and numerical modeling in general.
In Chapter~\ref{Getting Started}, we describe how to install \parflow{}, including
building the code and associated libraries. Then, we lead the user through a simple
\parflow{} run and discuss the automated test suite.  In
Chapter~\ref{The ParFlow System}, we describe the \parflow{} system in
more detail.  This chapter contains a lot of useful information regarding how a run is 
constructed and most importantly contains two detailed, annotated scripts that run two
classical \parflow{} problems, a fully saturated, heterogeneous aquifer and a variably 
saturated, transient, coupled watershed.  Both test cases are published in the literature
and are a terrific initial starting point for a new \parflow{} user.

Chapter~\ref{Manipulating Data} describes data analysis and processing. Chapter~\ref{Model_Equations} provides the basic equations solved
by \parflow{}.  Chapter~\ref{ParFlow Files} describes the formats of the
various files used by \parflow{}.  These chapters are really intended to be used as reference material. 
This manual provides some overview of \parflow{}
some information on building the code, examples of scripts that solve certain classes of
problems and a compendium of keys that are set for code options. 
